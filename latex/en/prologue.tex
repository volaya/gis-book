\chapter*{Prologue}

\vspace{-1cm}

When in 2005 I started writing my book ``Sistemas de Informaci\'{o}n Geogr\'{a}fica'', I did it for two reasons: first, because no books on GIS theory had been published in Spanish since the early 90's; second, because there were no free books about GIS, except those related to free GIS software, which included little theoretical content.

It took me five years to write the book, which ended up being a complete reference book with almost a thousand pages. Knowing that its size and its level of detail could be intimidating, and that many people would prefer a shorter version, in 2015, I wrote ``Introducci\'{o}n a los SIG''. The book you are reading now is the English translation of that shorter work.

Unlike what happens in Spanish, there are many good books on GIS theory written in English, and new editions are published constantly to update them with the latest changes in the field of GIS. However, no free book (that is, no book that can be freely copied, printed and distributed) on this topic had been published yet. 

I believe this book will be of great use for current GIS users and for anyone wanting to start in this fascinating field of GIS. If you have any suggestions or comments, you can contact me at: \texttt{volayaf@gmail.com}.